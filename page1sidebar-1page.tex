\cvsection{Projects}
%\divider

\cvproject{Face Detection using Faster R-CNN\hspace{0.3em}\href{https://github.com/anshulgupta0803/Face-Recognition-using-Faster-R-CNN}{\faGithub\hspace{0.3em}\faExternalLink}}{}%{Spring '17}
\begin{itemize}
	\item \textbf{OBJECTIVE} --- Draw bounding box on all the faces in an image
	\item \textbf{Focus}: Python, Image Processing
%	\item Developed a Faster-RCNN based model using VGG16 transfer learning
%	\item Training dataset (WIDER): 12K 600x600 images. Test dataset (FDDB): 2845 images with 5171 faces. The accuracy obtained was 87\%
\end{itemize}

%\divider

\cvproject{ASSR: Automatic Stuttered Speech Recognition\hspace{0.3em}\href{https://github.com/anshulgupta0803/ASSR}{\faGithub\hspace{0.3em}\faExternalLink}}{}%{Autumn '17}
\begin{itemize}
	\item \textbf{OBJECTIVE} --- Enable people having a stuttering speech impediment to use the current state-of-the-art speech-to-text systems
	\item \textbf{Focus}: Python, Audio Processing
%	\item UCLASS database was used for training and testing the ANN. IBM Watson's Speech-to-text system was used for validation of the corrected audio
%	\item Stuttered audio is segmented with the duration on 300ms and an overlap of 200ms which allowed the model to detect stutter at a granularity of 100ms
\end{itemize}

%\divider

\cvproject{Machine Learning approach for Music Genre Classification\hspace{0.3em}\href{https://github.com/anshulgupta0803/music-genre-classification}{\faGithub\hspace{0.3em}\faExternalLink}}{}%{Spring '16}
\begin{itemize}
	\item \textbf{OBJECTIVE} --- Classify music into different categories: Rock, Hip Hop, Jazz, Blues, etc.
	\item \textbf{Focus}: Python, Audio Processing
%	\item Feature set included Spectral Centroid, Spectral Rolloff, Time-domain Zero Crossing, MFCC coefficients, RMS Energy, and Spectral Contrast
%	\item Achieved an accuracy of 63.5\% using Random Forest. Tried a different approach of using CNN on the spectrogram of audio files. Accuracy was 23\%
\end{itemize}

%\divider

% \cvproject{Intelligent Reversi playing bot\hspace{0.3em}\href{https://github.com/anshulgupta0803/reversi}{\faGithub\hspace{0.3em}\faExternalLink}}{}%{Autumn '16}
% \begin{itemize}
% 	\item \textbf{OBJECTIVE} --- Build an intelligent agent to play Reversi
% 	\item \textbf{Focus}: Minimax Algorithm, Alpha-Beta Pruning
% %	\item Intelligence comes through Minimax algorithm with alpha-beta pruning
% %	\item The heuristic was a combination of parity, mobility, corners, and occupancy which determines the best possible next move
% \end{itemize}

%\divider

%\cvproject{Solving Sudoku using Boolean SAT Solver in Haskell\hspace{0.3em}\href{https://github.com/anshulgupta0803/SAT-Solver}{\faGithub\hspace{0.3em}\faExternalLink}}{}%{Autumn '17}
%\cvevent{}{Solving Sudoku using Boolean SAT Solver in Haskell}{Autumn '17}{}{\faGit\hspace{0.5em}\href{https://github.com/anshulgupta0803/SAT-Solver}{\faExternalLink}}
%\begin{itemize}
%	\item \textbf{OBJECTIVE} --- Build a SAT Solver and use it to solve Sudoku
%	\item \textbf{Focus}: Functional Programming
%	\item Implemented brute force and Davis-Putnam-Logemann-Loveland (DPLL) algorithms in Haskell%. DPLL reduces all the clauses in CNF as it progresses
%	\item Transformed Sudoku in 11,907 binary and 243 nine-ary CNF clauses which were fed to the SAT solver
%\end{itemize}

%\divider

\cvproject{Resource Provisioning of LXD Containers\hspace{0.3em}\href{https://github.com/anshulgupta0803/ContainerProvisioning}{\faGithub\hspace{0.3em}\faExternalLink}}{}%{Autumn '16}
\begin{itemize}
	\item \textbf{OBJECTIVE} --- Prevent memory SLA violations in LXD based containers
	\item \textbf{Focus}: Python, LXD Containers, Auto-Scaling
%	\item Implemented a framework which continuously monitors the containers in a cluster and alleviates memory hotspot by vertically and / or horizontally scaling the memory of containers
\end{itemize}

%\divider

%\cvproject{txt2midi: Indian musical notations to MIDI using Python}{Autumn '16}
%\begin{itemize}
%	\item \textbf{OBJECTIVE} --- Convert Indian classical music notation from text to MIDI
%	\item \textbf{Focus}: Python, MIDI
%%	\item Developed syntax for writing Indian classical music notation. Supports multiple instruments, mixing multiple tracks, setting volume and loop count for each track. The parser reads the input and generates a MIDI audio file
%\end{itemize}

\cvsection{Publications}
\nocite{*}
%\printbibliography[heading=none]

%\printbibliography[heading=pubtype,title={\printinfo{\faBook}{Books}},type=book]

%\divider

\printbibliography[heading=pubtype,title={\printinfo{\faFileTextO}{Journal Articles}}, type=article]

%\divider

\printbibliography[heading=pubtype,title={\printinfo{\faGroup}{Conference Proceedings}},type=inproceedings]

%\cvsection{Life Philosophy}
%\begin{quote}
%``If you don't have any shadows, you're not standing in the light.''
%\end{quote}


%\cvsection{Work Experience}
%
%\cvevent{Research Assistant}{CSE, IIT Bombay --- Prof. Varsha Apte}{Jul '16 --- Present}{}{}
%\begin{itemize}
%	\item Did experiments on client bottleneck detection and scalability of AutoPerf
%	\item Upgraded AutoPerf so that it supports newer Java version, Maven architecture, and Google's style guide
%\end{itemize}



%\cvsection{Languages}
%
%\cvskill{English}{5}
%% \divider
%
%\cvskill{Spanish}{4}
%% \divider
%
%\cvskill{German}{3}

% \vspace{0.4cm}
% \cvsection{Skills}
% \wheelchart{1.5cm}{0.5cm}{%
% 	20/3em/accent!80/\footnotesize\textbf{Python},
% 	4/3em/accent!16/\footnotesize\textbf{Java},
% 	20/3em/accent!80/\footnotesize\textbf{C++ / C},
% 	9/6em/accent!36/\footnotesize\textbf{Shell, awk \& sed},
% 	5/0em/accent!20/\footnotesize\textbf{\LaTeX},
% 	22/0em/accent!88/\footnotesize\textbf{Linux},
% 	8/0em/accent!32/\footnotesize\textbf{Git},
% 	12/6em/accent!48/\footnotesize\textbf{PHP / MySQL}
% }

\cvsection{Areas of Interest}
\cvtag{Computer Vision}
\cvtag{Medical Imaging}
% \cvtag{Speech Recognition}
%\cvtag{Data Mining}