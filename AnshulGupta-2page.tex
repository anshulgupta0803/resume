%% If you want to use \orcid or the
%% academicons icons, add "academicons"
%% to the \documentclass options.
%% Then compile with XeLaTeX or LuaLaTeX.
% \documentclass[10pt,a4paper,academicons]{altacv}

%% Use the "normalphoto" option if you want a normal photo instead of cropped to a circle
% \documentclass[10pt,a4paper,normalphoto]{altacv}

\documentclass[10pt,a4paper]{altacv}

% Change the page layout if you need to
\geometry{left=1cm,right=9cm,marginparwidth=6.8cm,marginparsep=1.2cm,top=1cm,bottom=1cm}

% Change the font if you want to.

% If using pdflatex:
\usepackage[utf8]{inputenc}
\usepackage[T1]{fontenc}
\usepackage[default]{lato}
\usepackage[colorlinks=true, urlcolor=black, urlbordercolor={0 1 1}]{hyperref}
\hypersetup{
	pdftitle={R\'esum\'e --- Anshul Gupta},
	pdfsubject={R\'esum\'e --- Anshul Gupta},
	pdfauthor={Anshul Gupta},
	pdfkeywords={CV,R\'esum\'e}
}

% If using xelatex or lualatex:
% \setmainfont{Lato}

% Change the colours if you want to
\definecolor{Black}{HTML}{000000}
\definecolor{SkyBlue}{HTML}{1796FE}
\definecolor{VividPurple}{HTML}{3E0097}
\definecolor{SlateGrey}{HTML}{2E2E2E}
\definecolor{LightGrey}{HTML}{666666}
\colorlet{heading}{Black}
\colorlet{accent}{SkyBlue}
\colorlet{emphasis}{SlateGrey}
\colorlet{body}{LightGrey}
\colorlet{heading}{Black}
\colorlet{accent}{Black}
%\colorlet{emphasis}{Black}
%\colorlet{body}{Black}

% Change the bullets for itemize and rating marker
% for \cvskill if you want to
\renewcommand{\itemmarker}{{\small\textbullet}}
\renewcommand{\ratingmarker}{\faCircle}

%% sample.bib contains your publications
\addbibresource{publications.bib}

\begin{document}
\name{Anshul Gupta}
\tagline{}
\photo{2.5cm}{me}
\personalinfo{
    \email{\href{mailto:anshulgupta0803@gmail.com}{anshulgupta0803@gmail.com}}% | \href{mailto:anshul@cse.iitb.ac.in}{anshul@cse.iitb.ac.in}}
    \phone{+91-916-662-8880}
%  \mailaddress{}
%  \location{}
    \homepage{\href{https://anshulgupta0803.github.io}{anshulgupta0803.github.io}}
%    \twitter{@anshulgupta0803}
%    \linkedin{anshulgupta0803}
%    \github{anshulgupta0803}
}

%% Make the header extend all the way to the right, if you want.
\begin{fullwidth}
	\makecvheader
	%\begin{minipage}{0pt}\vspace{165pt}\end{minipage}
\end{fullwidth}

%% Depending on your tastes, you may want to make fonts of itemize environments slightly smaller
\AtBeginEnvironment{itemize}{\small}

%% Provide the file name containing the sidebar contents as an optional parameter to \cvsection.
%% You can always just use \marginpar{...} if you do
%% not need to align the top of the contents to any
%% \cvsection title in the "main" bar.
\cvsection[page1sidebar-2page]{Internships}

\cvevent{Google Summer of Code 2017 (The Linux Foundation)}{Common Print Dialog}{Jun '17 --- Aug '17}{---}{\faGit\hspace{0.5em}\href{https://anshulgupta0803.github.io/common-print-dialog/}{\faExternalLink}}
\begin{itemize}
	\item \textbf{OBJECTIVE} --- Build a unified solution for printing in desktop environments. A well designed print dialog will help the users to find the right printers and printing configurations
	\item Developed an ergonomic front-end written in Qt as a part of 5-strong team. It communicates with the back-end using DBus which supports printing with CUPS, IPP or Google Cloud Print
\end{itemize}

\divider

%\cvevent{Ecole Normale Sup\'erieure de Lyon (INRIA)}{Analysis and Extrapolation of CPU Processing Rate for the Simulation of Parallel Applications}{May '14 --- Jul '14}{Lyon, France}{\faGit\hspace{0.5em}\href{https://github.com/anshulgupta0803/scalatrace-ti}{\faExternalLink}}
%\cvevent{Laboratoire de l'Informatique du Parall\'elisme (ENS de Lyon)}{Towards more scalable off-line simulation of MPI applications}{May '14 --- Jul '14}{Lyon, France}{\faGit\hspace{0.5em}\href{https://github.com/anshulgupta0803/scalatrace-ti}{\faExternalLink}}
\cvevent{\'Ecole Normale Sup\'erieure de Lyon (INRIA)}{Towards more scalable off-line simulation of MPI applications}{May '14 --- Jul '14}{Lyon, France}{\faGit\hspace{0.5em}\href{https://github.com/anshulgupta0803/scalatrace-ti}{\faExternalLink}}
\begin{itemize}
	\item \textbf{OBJECTIVE} --- Build a framework for scalable time-independent trace replay for off-line simulation of MPI applications
	\item Merged ScalaTrace and Time-Independent Trace Replay to run simulations using Simgrid. Used NAS Parallel Benchmark to validate the framework
\end{itemize}

\divider

\cvevent{Indraprastha Institute of Information Technology, Delhi}{Performance analysis and Optimization of Hadoop based cluster}{May '13 --- Jul '13}{New Delhi, India}{}
\begin{itemize}
	\item \textbf{OBJECTIVE} --- Analyze the effect of various configuration parameters on Hadoop Map-Reduce performance under various conditions to achieve maximum throughput
	\item Studied the effect of block-size, copy phase, map spill and reduce phase for Fair, Capacity, and FCFS Scheduler on the throughput and execution time
\end{itemize}

\cvsection{M.Tech. Dissertation}
\cvevent{}{Scalability, Reliability, and Security of BodhiTree}{Jul '18 --- Present}{}{}
\begin{itemize}
	\item \textbf{OBJECTIVE} --- Supporting no more than 250 students combined with the security bugs, BodhiTree needs to be revamped, so that theoretically, its performance can increase linearly with linear increase in resources
	%\item BodhiTree - an initiative for e-learning by IIT Bombay mimics a classroom setting, providing flexibility for students to study at their own pace
	\item Expected outcome from this project is a scalable LMS which can be distributed (as an application container) and deployed with ease
\end{itemize}

\divider

\cvevent{}{Modeling Virtualized Applications using Machine Learning Techniques}{Spring 2017}{}{}
\begin{itemize}
	\item Performance models allows administrators to explore "what-if?" scenarios without the need of actual hardware. Simulations can be performed to get the expected performance if the model is correct
	\item Surveyed various machine learning approaches like ANNs/SVMs, Kalman Filters, Markov Models and Self-Organizing Maps for building performance models of virtualized applications
\end{itemize}

\clearpage

\cvsection[page2sidebar-2page]{Projects}
\cvevent{}{ASSR: Automatic Stuttered Speech Recognition}{Autumn '17}{}{\faGit\hspace{0.5em}\href{https://github.com/anshulgupta0803/ASSR}{\faExternalLink}}
\begin{itemize}
	\item \textbf{OBJECTIVE} --- Enabling people having a stuttering speech impediment to use the current state-of-the-art speech-to-text systems
	\item UCLASS database was used for training and testing the ANN. IBM Watson's Speech-to-text system was used for validation of the corrected audio
%	\item Stuttered audio is segmented with the duration on 300ms and an overlap of 200ms which allowed the model to detect stutter at a granularity of 100ms
\end{itemize}

\divider

\cvevent{}{Face Recognition using Faster R-CNN}{Spring '17}{}{\faGit\hspace{0.5em}\href{https://github.com/SeeTheC/Face-Detection-using-Faster-R-CNN}{\faExternalLink}}
\begin{itemize}
	\item \textbf{OBJECTIVE} --- Draw bounding box on all the faces in an image
	\item Developed a Faster-RCNN based model using VGG16 transfer learning
	\item Training dataset (WIDER): 12K 600x600 images. Test dataset (FDDB): 2845 images with 5171 faces. The accuracy obtained was 87\%
\end{itemize}

\divider

\cvevent{}{Machine Learning approach for Music Genre Classification}{Spring '16}{}{\faGit\hspace{0.5em}\href{https://github.com/anshulgupta0803/music-genre-classification}{\faExternalLink}}
\begin{itemize}
	\item \textbf{OBJECTIVE} --- Classify music into different categories: Rock, Hip Hop, Jazz, Metal, Classical, Pop, Disco, Blues, Reggae, Country
%	\item Feature set included Spectral Centroid, Spectral Rolloff, Time-domain Zero Crossing, MFCC coefficients, RMS Energy, and Spectral Contrast
	\item Achieved an accuracy of 63.5\% using Random Forest. Tried a different approach of using CNN on the spectrogram of audio files. Accuracy was 23\%
\end{itemize}

\divider

\cvevent{}{Intelligent Reversi playing bot}{Autumn '16}{}{\faGit\hspace{0.5em}\href{https://github.com/anshulgupta0803/reversi}{\faExternalLink}}
\begin{itemize}
	\item \textbf{OBJECTIVE} --- Build an intelligent agent to play Reversi
	\item Intelligence comes through Minimax algorithm with alpha-beta pruning
	\item The heuristic was a combination of parity, mobility, corners, and occupancy which determines the best possible next move
\end{itemize}

\divider

\cvevent{}{Solving Sudoku using Boolean SAT Solver in Haskell}{Autumn '17}{}{\faGit\hspace{0.5em}\href{https://github.com/anshulgupta0803/SAT-Solver}{\faExternalLink}}
\begin{itemize}
	\item \textbf{OBJECTIVE} --- Build a SAT Solver and use it to solve Sudoku
	\item Implemented brute force and Davis-Putnam-Logemann-Loveland (DPLL) algorithms in Haskell%. DPLL reduces all the clauses in CNF as it progresses
	\item Transformed Sudoku in 11,907 binary and 243 nine-ary CNF clauses which were fed to the SAT solver
\end{itemize}

\divider

\cvevent{}{Resource Provisioning of LXD Containers}{Autumn '16}{}{\faGit\hspace{0.5em}\href{https://github.com/anshulgupta0803/ContainerProvisioning}{\faExternalLink}}
\begin{itemize}
	\item \textbf{OBJECTIVE} --- Prevent memory SLA violations in LXD based containers
	\item Implemented a framework which continuously monitors the containers in a cluster and alleviates memory hotspot by vertically and / or horizontally scaling the memory of containers
\end{itemize}

\divider

\cvevent{}{txt2midi: Indian musical notations to MIDI using Python}{Autumn '16}{}{\faGit\hspace{0.5em}\href{https://github.com/anshulgupta0803/txt2midi}{\faExternalLink}}
\begin{itemize}
	\item \textbf{OBJECTIVE} --- Convert Indian classical music notation from text to MIDI
	\item Developed syntax for writing Indian classical music notation. Supports multiple instruments, mixing multiple tracks, setting volume and loop count for each track. The parser reads the input and generates a MIDI audio file
\end{itemize}

%% If the NEXT page doesn't start with a \cvsection but you'd
%% still like to add a sidebar, then use this command on THIS
%% page to add it. The optional argument lets you pull up the
%% sidebar a bit so that it looks aligned with the top of the
%% main column.
% \addnextpagesidebar[-1ex]{page3sidebar}


\end{document}
