%% If you want to use \orcid or the
%% academicons icons, add "academicons"
%% to the \documentclass options.
%% Then compile with XeLaTeX or LuaLaTeX.
% \documentclass[10pt,a4paper,academicons]{altacv}

%% Use the "normalphoto" option if you want a normal photo instead of cropped to a circle
% \documentclass[10pt,a4paper,normalphoto]{altacv}

\documentclass[10pt,a4paper]{altacv}

% Change the page layout if you need to
\geometry{left=1cm,right=9cm,marginparwidth=6.8cm,marginparsep=1.2cm,top=1cm,bottom=1cm}

% Change the font if you want to.

% If using pdflatex:
\usepackage[utf8]{inputenc}
\usepackage[T1]{fontenc}
\usepackage[default]{lato}
\usepackage[colorlinks=true, urlcolor=black, urlbordercolor={0 1 1}]{hyperref}
\hypersetup{
	pdftitle={R\'esum\'e --- Anshul Gupta},
	pdfsubject={R\'esum\'e --- Anshul Gupta},
	pdfauthor={Anshul Gupta},
	pdfkeywords={CV,R\'esum\'e}
}

% If using xelatex or lualatex:
% \setmainfont{Lato}

% Change the colours if you want to
\definecolor{Black}{HTML}{000000}
\definecolor{SkyBlue}{HTML}{1796FE}
\definecolor{VividPurple}{HTML}{3E0097}
\definecolor{SlateGrey}{HTML}{2E2E2E}
\definecolor{LightGrey}{HTML}{666666}
%\colorlet{heading}{Black}
%\colorlet{accent}{SkyBlue}
\colorlet{emphasis}{SlateGrey}
\colorlet{body}{LightGrey}
\colorlet{heading}{Black}
\colorlet{accent}{Black}
%\colorlet{emphasis}{Black}
%\colorlet{body}{Black}

% Change the bullets for itemize and rating marker
% for \cvskill if you want to
\renewcommand{\itemmarker}{{\small\textbullet}}
\renewcommand{\ratingmarker}{\faCircle}

%% sample.bib contains your publications
\addbibresource{publications.bib}

\begin{document}
\name{Anshul Gupta}
\tagline{}
%\photo{2.0cm}{me}
\personalinfo{
    \email{\href{mailto:anshulgupta0803@gmail.com}{anshulgupta0803@gmail.com}}% | \href{mailto:anshul@cse.iitb.ac.in}{anshul@cse.iitb.ac.in}}
    \phone{+91 916 662 8880}
%  \mailaddress{}
%  \location{}
    \homepage{\href{https://bit.ly/ansh0803}{bit.ly/ansh0803}}
%    \twitter{@anshulgupta0803}
%    \linkedin{anshulgupta0803}
%    \github{anshulgupta0803}
}

%% Make the header extend all the way to the right, if you want.
\begin{fullwidth}
	\makecvheader
	%\begin{minipage}{0pt}\vspace{165pt}\end{minipage}
\end{fullwidth}

%% Depending on your tastes, you may want to make fonts of itemize environments slightly smaller
\AtBeginEnvironment{itemize}{\small}

%% Provide the file name containing the sidebar contents as an optional parameter to \cvsection.
%% You can always just use \marginpar{...} if you do
%% not need to align the top of the contents to any
%% \cvsection title in the "main" bar.
\cvsection[page1sidebar-1page]{Experience}
\cvevent{Entrepreneur First}{Founder in Residence}{Feb '21 --- Present}{Bangalore, India}{}
\begin{itemize}
    \item Using computer vision to increase success rates of In Vitro Fertilization (IVF)
    \item Hypothesis validation and customer development for the product
	\item \textbf{Focus}: Entrepreneurship, Healthcare, Computer Vision
\end{itemize}

\divider

\cvevent{Samsung R\&D Institute India --- Bangalore}{Senior Software Engineer}{Jul '19 --- Jan '21}{Bangalore, India}{}
\begin{itemize}
    \item Super Resolution for 3D Ultrasound images of fetal face
    \item Follicle Growth Tracking in ovaries using 3D Ultrasound Scans
	\item Identity network based on UNet model for medical images to be used as a better initialization for transfer learning 
	\item \textbf{Focus}: Medical Imaging, Computer Vision
\end{itemize}

% \divider

\cvsection{Internships}
\cvevent{Google Summer of Code 2017 (The Linux Foundation)}{Common Print Dialog}{Jun '17 --- Aug '17}{---}{\faGithub\hspace{0.5em}\href{https://anshulgupta0803.github.io/common-print-dialog/}{\faExternalLink}}
\begin{itemize}
    % \item Project: \textbf{Common Print Dialog}
	\item Designed and built a unified solution for printing in desktop environments
	\item \textbf{Focus}: C++, Qt, DBus
\end{itemize}

\divider

\cvevent{\'Ecole Normale Sup\'erieure de Lyon (INRIA)}{Towards more scalable off-line simulation of MPI applications}{May '14 --- Jul '14}{Lyon, France}{\faGithub\hspace{0.5em}\href{https://github.com/anshulgupta0803/scalatrace-ti}{\faExternalLink}}
\begin{itemize}
    % \item Project: \textbf{Towards more scalable off-line simulation of MPI applications}
	\item Developed a framework for scalable time-independent trace replay for off-line simulation of MPI applications
	\item \textbf{Focus}: R, MPI, C++
%	\item Merged ScalaTrace and Time-Independent Trace Replay to run simulations using Simgrid. Used NAS Parallel Benchmark to validate the framework
\end{itemize}

\divider

\cvevent{Indraprastha Institute of Information Technology, Delhi}{Performance analysis and Optimization of Hadoop based cluster}{May '13 --- Jul '13}{New Delhi, India}{}
\begin{itemize}
    % \item Project: \textbf{Performance analysis and Optimization of Hadoop based cluster}
	\item Analyzed the effect of various configuration parameters on Hadoop performance under various conditions to achieve maximum throughput
	\item \textbf{Focus}: Hadoop
%	\item Studied the effect of block-size, copy phase, map spill and reduce phase for Fair, Capacity, and FCFS Scheduler on the throughput and execution time
\end{itemize}


\cvsection{M.Tech. Dissertation}
\cvevent{}{Scalability, Reliability, and Security of BodhiTree}{Jul '18 --- Jun '19}{}{}
\begin{itemize}
    \item Worked on an in-house learning management system used by students and faculties
    \item Improved throughput of system from 8 req/s to 154 req/s
    \item Dockerized entire platform for ease of deployment and hardened security
    \item Load balancing for the entire platform using HAProxy and added support for horizontal and vertical scaling
	%\item \textbf{OBJECTIVE} --- Supporting no more than 250 students combined with the security bugs, BodhiTree needs to be revamped, so that theoretically, its performance can increase linearly with linear increase in resources
	%\item BodhiTree - an initiative for e-learning by IIT Bombay mimics a classroom setting, providing flexibility for students to study at their own pace
	%\item Expected outcome from this project is a scalable LMS which can be distributed (as an application container) and deployed with ease
\end{itemize}

%% If the NEXT page doesn't start with a \cvsection but you'd
%% still like to add a sidebar, then use this command on THIS
%% page to add it. The optional argument lets you pull up the
%% sidebar a bit so that it looks aligned with the top of the
%% main column.
% \addnextpagesidebar[-1ex]{page3sidebar}


\end{document}
